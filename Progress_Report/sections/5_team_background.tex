\section{Team Background}
For this project it relates to Drake’s degree of computer systems engineering very heavily as it will be relying on arduinos, and various sensors to observe muscle movements of the skater, accelerometers to track the speed and pressure sensors to see how well the skater is skating. The project relates to the program on the software side as well with the setting up of these sensors and potentially sending the data to cloud based data collectors.

The project relates to Kate’s program of Biomedical and Electrical Engineering. Various aspects within the project relate to courses that she has taken throughout her years at Carleton University. The biomedical aspect of her program relates to the project due to the movements of various muscles that will be analyzed using EMG muscle sensors, as well as the pressure sensors and the placements of them. As well, there are various electrical aspects relating to her program that will be used to achieve the objective such as circuit building and usage of an Arduino Uno.

This project is related to Mario’s program in Computer Systems Engineering because of the importance of combining the usage of computer hardware with software in data gathering. The program involves learning a mixture of hardware and software and the interactions between the physical realm of hardware and the digital realm that software inhabits and their communication. 

This project relates to Braden’s program of Computer Systems Engineering as in this project there will be a very prominent software component. In Braden’s program there is a great emphasis on the software side of projects. He has learned many different coding languages (one being python extensively) and has also worked with a variety of boards (raspberry pi and arduino) as well. Because of this, the arduinos that will be used in this project are ideal for him. In Braden’s 3rd year project (Sysc 3010) he had to use multiple sensors including a sound sensor, motion sensor and a photoresistor, so he has experience with various sensors as well. 

This project relates to Marko’s program of Software Engineering as it encompasses many aspects of the software development cycle as well as various types of software development. The project includes embedded systems development where software used to control the arduino microcontroller and all accompanying hardware needs to be developed and tested. The project includes real-time concurrent systems development, and database management  as the arduino and python application will be concurrently communicating with the database in real-time. Then a python application with a back-end and front-end that will pull the data, parse and analyze it, and display it in an intelligible way to the end-user in real-time also needs to be developed and tested. Further, all the software being developed will need to go through the requirements engineering process, architecture and design, and validation and verification - all of which has been covered during his program.  Thus, this project provides applied exposure to different types of software development for different purposes, as well as the opportunity to exercise the software development lifecycle.

This project relates to Connor’s Program of Electrical Engineering through its use of microcontrollers and sensors. Connor has experience with microcontrollers from his third year project and the ELEC 4601 class. This project will allow for an expansion on the understanding and utilization of microcontrollers and their sensors. In addition the programming language for the arduino microcontroller is C++ which was taught in ECOR 1606. This project will increase the understanding of the programming language.
