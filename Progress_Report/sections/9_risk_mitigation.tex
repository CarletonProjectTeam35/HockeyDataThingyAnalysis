\section{Risk Mitigation}
Due to the nature of the project there are a variety of risk factors that the team has acknowledged and will be designing the project tests around. The project involves ice hockey and roller blading and due to this, any test subjects will be required to be wearing proper safety equipment to minimize injuries while on the ice or flooring. Outside of the usage of proper safety equipment the subjects that will be used were chosen due to their experience with skating. As only the skating element of hockey is being analyzed there will not be physical contact or interference of skating by hockey sticks or hockey pucks/balls.  Furthermore, by using indoor hockey rinks some of the safety risks that are present with outdoor ice hockey are avoided. This includes less well kept ice and highly decreases the risk of hardware malfunction due to temperature and weather. The roller blading being done in a gymnasium will also mitigate some of the risks that rollerblading outdoors on a road or cement would have. While the data being collected is done at a controlled location separate from anyone’s house, mitigations of privacy risk will be conducted by not identifying in the software or report whose data was being gathered at any given time.

To mitigate the risk of inadequate datasets, and get a representative model a large sample of data will be collected from as many as 5 test subjects. 

To address the risk of software or hardware defects, stringent testing will be put in place both during the development process, as well as once the system is complete and ready to use, but prior to data collection.
