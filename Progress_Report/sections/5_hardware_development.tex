\subsection{Hardware Development}
\subsubsection{Variations from initial plans}
There have been some minor variations to the original plan this being the implementation of the force sensors. In order for the team to have used these we would have to make a circuit on the bread-board with and op-amp as the original force sensor could only handle up to 20kg of force. At first this did not seem like a problem, but as the implementation went on we realized it would not give us an accurate reading as the skater may exceed these forces and that it would not be viable having an breadboard and op-amps on the skater during testing. These caused the switched to what the team is now using which are load cells, these can carry up to 400kg when wired in series and can be calibrated to each skater giving the team lots of room to work with. The team will have eight load cells sensors in total with four in each foot, offering 400kg of force to be collected from the skater. 
\subsubsection{Progress}
The team has made substantial progress in the way of hardware, having completed the soldering of six EMG sensors and there required pins on the board and wires that will then connect to the arduino, as shown in the appendix. The arduinio is then coded and the data is sent to our thingspeak channel, where it will then be read by the software side of the project. The load cell sensors were all wired up in series, as there was four per series, we had to solder up eight load cell sensors in total as well as there amplifier module that needed the required pins to then be hooked up the arduino. This can be seen in the appendix 1.2. The gyroscope is built into the board and has been setup to record the change of the boards position overtime.   
