\subsection{Hardware Development}
\subsubsection{Variations from initial plans}
As mentioned in section 4.1.1, in order to have used the force sensitive resistors, the team would have had to make expand the circuitry on board with op-amps to permit readings greater than 20kg of force. At first this did not seem like a problem, but as the implementation went on the team realized this would introduce a large margin of error, and create for unreliable readings. For this reason the team made a switch to laod sensors which can measure up to 400kg when wired in series and can be calibrated to each skater giving. The team will have eight load sensors in total with four in each foot, offering 400kg of force to be collected from the skater, and providing for far more accurate results.
\subsubsection{Progress}
The team has made substantial development on the hardware, having completed the soldering of six EMG sensors into their required pins on the board and wires that will then connect to the arduino, as shown in the appendix. The arduinio has been programmed to send data to our ThingSpeak channel, where it will then be read by the software side of the project. The load cell sensors were all wired up in series, as there was four per series, the team had to solder eight load cell sensors in total as well as their amplifier module that needed the required pins to then be hooked up the arduino. This can be seen in the appendix 1.2. They have been finished and have been placed in soles of a shoe for easy access as the team will be switching participants frequently, and this will allow for a modular setup. The gyroscope is built into the board and has been setup to record the change of the boards position overtime.   
