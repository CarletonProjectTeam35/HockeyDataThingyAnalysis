\subsection{Group’s Collective Ability to Accomplish Project Objectives}

The project group, with six total members, contains a wide variety of technical backgrounds - 3 Computer Systems engineering students, 1 Software Engineering Student, 1 Biomedical Electrical Engineering Student, and 1 Electrical Engineering Student. The project being pursued spans multiple technical fields in which assets with specialization in many fields are required to successfully accomplish the objectives of the project. The physical construction, operation, and interfacing of complex hardware components require the expertise gained in an Electrical Engineering Degree. However, determining use cases for the hardware components, proper application of hardware components, and interpretation of collected data require knowledge of human biology in the context of electrical engineering in which the expertise gained in Biomedical Electrical degree is an asset. In order to develop the critical embedded software to run the hardware and collect data requires expertise in Computer Systems Engineering, and Software Engineering. When the data is collected, it needs to be efficiently and effectively communicated wirelessly to the database, pulled from the database, analyzed and outputted in an intelligible format all in real-time, which requires, again, expertise acquired in Computer Systems Engineering, and Software Engineering. Additionally, the software needs to be put through stringent testing and quality analysis to ensure its soundness which requires knowledge acquired in Software engineering. With members of the team covering all these needed technical areas, the team has a robust foundation on all fronts, and a proven exceptional ability to collaborate with one another, which is a strong indication that the team is capable of successfully carrying out the project to completion. 